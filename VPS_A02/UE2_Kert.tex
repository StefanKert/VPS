\documentclass[a4paper,ngerman]{scrartcl}

\usepackage[utf8]{inputenc}
\usepackage[T1]{fontenc}
\usepackage{babel}

\usepackage{paralist}
\usepackage{listings} 
\usepackage{datetime} 
\usepackage{graphicx}
\usepackage{enumitem}
\usepackage{booktabs}

\usepackage{color}
 
\definecolor{bluekeywords}{rgb}{0,0,1}
\definecolor{greencomments}{rgb}{0,0.5,0}
\definecolor{redstrings}{rgb}{0.64,0.08,0.08}
\definecolor{xmlcomments}{rgb}{0.5,0.5,0.5}
\definecolor{types}{rgb}{0.17,0.57,0.68}
 
\lstset{language=[Sharp]C,
captionpos=b,
showspaces=false,
showtabs=false,
breaklines=true,
showstringspaces=false,
breakatwhitespace=true,
escapeinside={(*@}{@*)},
commentstyle=\color{greencomments},
morekeywords={partial, var, value, get, set},
keywordstyle=\color{bluekeywords},
stringstyle=\color{redstrings},
basicstyle=\ttfamily\footnotesize,            
tabsize=2
}
 

\begin{document}

\title{VPS5 - UE2}
\author{Stefan Kert}
\date{\today}
\maketitle

\section{Race Conditions}
\subsection{Was sind \textit{Race Conditions}?}
Eine Race Condition tritt dann auf, wenn zwei Threads gleichzeitig versuchen, eine Operation auf einem Objekt durchzuführen, wodurch sich ein falsches Ergebnis ergibt.


\subsection{Was kann getan werden um \textit{Race Conditions} zu vermeiden?}

\end{document}

\end{document}
