\documentclass[a4paper,ngerman]{scrartcl}

\usepackage[utf8]{inputenc}
\usepackage[T1]{fontenc}
\usepackage{babel}

\usepackage{paralist}
\usepackage{listings} 
\usepackage{datetime} 
\usepackage{graphicx}
\usepackage{enumitem}
\usepackage{booktabs}

\usepackage{color}
 
\definecolor{bluekeywords}{rgb}{0,0,1}
\definecolor{greencomments}{rgb}{0,0.5,0}
\definecolor{redstrings}{rgb}{0.64,0.08,0.08}
\definecolor{xmlcomments}{rgb}{0.5,0.5,0.5}
\definecolor{types}{rgb}{0.17,0.57,0.68}
 
\lstset{language=[Sharp]C,
captionpos=b,
showspaces=false,
showtabs=false,
breaklines=true,
showstringspaces=false,
breakatwhitespace=true,
escapeinside={(*@}{@*)},
commentstyle=\color{greencomments},
morekeywords={partial, var, value, get, set},
keywordstyle=\color{bluekeywords},
stringstyle=\color{redstrings},
basicstyle=\ttfamily\footnotesize,            
tabsize=2
}
 

\begin{document}

\title{VPS5 - UE3}
\author{Stefan Kert}
\date{\today}
\maketitle

\section{Dish of the Day: Almondbreads}
\subsection{a. Sync - Generator}
Der Synchrone Generator wurde bereits in der Übungsstunde implementiert und nur leicht abgeändert, sodass die weiteren Komponenten die für diese Übung entwickelt wurden den Code dieses Generators verwenden können. Dabei enthält der SyncImageGenerator relativ wenig Code, da dort nur die Logik für die Stopwatch bzw. das Aufrufen des Events und der \textit{GenerateBitmap} Methode getätigt wird.

\begin{lstlisting}
    public class SyncImageGenerator : AbstractImageGenerator
    {
        public override void GenerateImage(Area area) {
            var stopwatch = new Stopwatch();
            stopwatch.Start();
            var bm = GenerateBitmap(area, Source.Token);
            stopwatch.Stop();
            OnImageGenerated(area, bm, stopwatch.Elapsed);
        }
    }
\end{lstlisting}

Im \textit{AbstractImageGenerator}, der als Basisklasse für sämtliche Generatoren dient, ist die Logik für das generieren der einzelnen Zeilen und Spalten des Bitmaps, sowie etwaige Events und Hilfsmethoden. 

\begin{lstlisting}
     public abstract class AbstractImageGenerator 
    {
        protected CancellationTokenSource Source;

        protected AbstractImageGenerator(): this(new CancellationTokenSource()) {}
        private AbstractImageGenerator(CancellationTokenSource source) {
            Source = source;
        }

        protected virtual Bitmap GenerateBitmap(Area area, CancellationToken cancelToken)
        {
            Bitmap bitmap = new Bitmap(area.Width, area.Height);
            for (int column = 0; column < area.Width; column++) {
                for (int row = 0; row < area.Height; row++) {
                    if (cancelToken.IsCancellationRequested)
                        return null;
                    bitmap.SetPixel(column, row, ImageGeneratorLogic.GetColorForData(area, column, row, cancelToken));
                }
            }
            return bitmap;
        }



        public abstract void GenerateImage(Area area);

        public event EventHandler<EventArgs<Tuple<Area, Bitmap, TimeSpan>>> ImageGenerated;

        protected void OnImageGenerated(Area area, Bitmap bitmap, TimeSpan timeSpan) {
            if (area != null && bitmap != null)
                ImageGenerated?.Invoke(this, new EventArgs<Tuple<Area, Bitmap, TimeSpan>>(new Tuple<Area, Bitmap, TimeSpan>(area, bitmap, timeSpan)));
        }
    }
\end{lstlisting}

Um eine bessere Auftrennung der einzelnen Aspekte zu gewährleisten, wurde die Methode für die Generierung der Farbe der einzelnen Felder in eine eigene Klasse \textit{ImageGeneratorLogic} ausgelagert. Dies dient in erster Linie dem Code Reuse, sowie der Testbarkeit, da durch ein herauslösen dieser Methode eine leichtere Testbarkeit gewährleistet ist.


\begin{lstlisting}
       public static class ImageGeneratorLogic
    {
        private static int MaxIterations => Settings.DefaultSettings.MaxIterations;
        private static double Border => Settings.DefaultSettings.ZBorder * Settings.DefaultSettings.ZBorder;

        public static Color GetColorForData(Area area, int column, int row, CancellationToken cancelToken)
        {
            var cReal = area.MinReal + column * area.PixelWidth;
            var cImg = area.MinImg + row * area.PixelHeight;
            var zReal = 0.0;
            var zImg = 0.0;

            var k = 0;
            while ((zReal * zReal + zImg * zImg < Border) && (k < MaxIterations))
            {
                if (cancelToken.IsCancellationRequested)
                    return Color.Empty;
                var zNewReal = zReal * zReal - zImg * zImg + cReal;
                var zNewImg = 2 * zReal * zImg + cImg;

                zReal = zNewReal;
                zImg = zNewImg;

                k++;
            }
            return ColorSchema.GetColor(k);
        }
    }
\end{lstlisting}

\subsection{b. Async- Generator}
Um Userinteraktionen auch während einer Berechnung zu ermöglichen ist es nötig, die Berechnung des Bildes asynchron auszuführen. Dazu wurden bereits für den Synchronen Generator Vorkehrungen getroffen. 
Diese Vorkehrungen waren in erster Linie das Hinzufügen eines Events, welches signalisieren sollte, dass die Berechnung abgeschlossen wurde. Diese Event wird über die Methode \textit{OnImageGenerated} aufgerufen. Das Auslösen dieses Events ruft eine Methode im der \textit{MainForm} auf, welche schließlich das Bild lädt und die Informationen über die benötigte Zeit setzt. Hier ist nur wichtig, dass darauf geachtet wird, dass die Aktionen in der GUI im GUI Thread getätigt werden. Hierzu ist eine Synchronisierung notwendig. 

In den beiden Implementierungen sollte darauf geachtet werden, dass falls eine erneute Berechnung gestartet wird, eine möglicherweise laufende Berechnung beendet wird. Dies erfolgt jeweils mittels Speichern der jeweiligen Thread, bzw. Backgroundworker Instanz in einer Membervariable und durch die Überprüfung dieser. Nähere Informationen hierzu in den einzelnen Unterpunkten dieser Implementierungen. Da die Asynchrone Variante mit \textit{CancellationTokens} arbeitet, muss hier darauf geachtet werden, dass dieser bei einem Abbrechen gecanceld wird und gewartet wird bis dieses Canceln abgeschlossen wurde.  Danach kann die jeweilige Instanz neu erstellt werden und eine neue Berechnung gestartet werden.


\subsubsection{AsyncGenerator mit Threads}
Die erste Implementierung wurde mit einfachen Threads realisiert. Dazu wird ein Thread mit der Methode gestartet, welche die Berechnung vornimmt. Heir wird überprüft ob die Membervariable ungleich null ist und ob das Flag \textit{IsAlive} true ist. Wenn dies der Fall ist wird über den \textit{Cancellationtoken} die Operation abgebrochen und mit\textit{ WaitOne()} gewartet bis dieses Abbrechen abgeschlossen ist. Danach wird eine neue Instanz des \textit{CancellationTokenSource} erstellt und die neue Berechnung gestartet. Wenn die Berechnung abgeschlossen wird, wird die Methode \textit{OnImageGenerated} aufgerufen, welche wie oben beschrieben signalisiert, dass die Operation beendet worden ist.

\begin{lstlisting}
    public class AsyncImageGenerator : AbstractImageGenerator
    {
        private Thread _calculationThread;

        public override void GenerateImage(Area area) {
            if (_calculationThread != null && _calculationThread.IsAlive) {
                Source.Cancel();
                Source.Token.WaitHandle.WaitOne();
                Source = new CancellationTokenSource();
            }
            _calculationThread = new Thread(() => BuildBitmap(area));
            _calculationThread.Start();
        }

        protected virtual void BuildBitmap(Area area) {
            var stopwatch = new Stopwatch();
            stopwatch.Start();
            var bm = GenerateBitmap(area, Source.Token);
            stopwatch.Stop();
            OnImageGenerated(area, bm, stopwatch.Elapsed);
        }
    }
\end{lstlisting}

\subsubsection{AsyncGenerator mit Backgroundworker}
Die Implementierung mittels \textit{Backgroundworker} funktioniert sehr ähnlich. Auch hier wird mittels einer Membervariable auf null überprüft, und ob der \textit{Backgroundworker} gerade läuft. Wenn ja wird wiederum über den \textit{CancellationToken} ein \textit{Cancel()} aufgerufen und die Operation somit beendet. Auch hier wird gewartet und nach Abschluss wird der \textit{BackgroundWorker} gecancelt. Es wäre hier auch möglich gewesen, die CancelationEvents über den \textit{BackgroundWorker} zu implementieren, jedoch würde dies zu einer Codeduplizierung führen, da die Logik für die Cancellation erneut implementiert werden müsste. Beim \textit{Backgroundworker} werden die Events \textit{DoWork} und \textit{RunWorkerCompleted} gebunden. Über \textit{DoWork} wird die Methode zum Generieren des Bitmaps gestartet. Nach Abschluss dieser wird der Result Member der \textit{DoWorkEventArgs} gesetzt welcher schließlich in der \textit{RunWorkerCompleted} Methode verwendet wird. In dieser Methode wird die \textit{OnImageGenerated} Methode aufgerufen womit die Berechnung beendet ist.

\begin{lstlisting}
    public class AsyncImageGeneratorWithBackgroundWorker : AbstractImageGenerator
    {
        private BackgroundWorker _bw;

        public override void GenerateImage(Area area) {
            if (_bw != null && _bw.IsBusy) {
                Source.Cancel();
                Source.Token.WaitHandle.WaitOne();
                Source = new CancellationTokenSource();
                _bw.CancelAsync();
            }

            _bw = new BackgroundWorker {WorkerSupportsCancellation = true};
            _bw.DoWork += (sender, args) => BuildBitmap(area, args);
            _bw.RunWorkerCompleted += OnBuildBitmapCompleted;
            _bw.RunWorkerAsync();
        }

        private void BuildBitmap(Area area, DoWorkEventArgs args)
        {
            var stopwatch = new Stopwatch();
            stopwatch.Start();
            var bm = GenerateBitmap(area, Source.Token);
            stopwatch.Stop();
            args.Result = new Tuple<Area, Bitmap, TimeSpan>(area, bm, stopwatch.Elapsed);
        }

        private void OnBuildBitmapCompleted(object sender, RunWorkerCompletedEventArgs e) {
            var bw = sender as BackgroundWorker;
            if (bw != null)
                bw.RunWorkerCompleted -= OnBuildBitmapCompleted;
            var res = e.Result as Tuple<Area, Bitmap, TimeSpan>;
            OnImageGenerated(res?.Item1, res?.Item2, res?.Item3 ?? new TimeSpan());
        }
    }
\end{lstlisting}

\subsection{c. Parallel - Generator}
Durch ein Aufteilen der Berechnung in mehrere Threads und ein paralleles ausführen dieser Berechnungen kann ein Speed Up erreicht werden. Dazu werden mehrere Threads erzeugt und jedem dieser Threads wird eine Methode zur Berechnung einer Zeile zugeordnet. Die Nummer der Zeile werden über eine Membervariable ausgelesen, welche auch über einen \textit{lock} synchronisiert wird. Diese Zeilennummer gibt aufschluss darüber, welche Reihe gerade berechnet wird. Nach der Berechnung der aktuellen Farbe wird das bitmap gelockt und die Pixel mit der jeweiligen Farbe gesetzt. Nach dem Initialisieren der Threads wird mittels Schleife auf das Enden der einzelnen Threads gewartet und schließlich, insofern die Operation nicht vorher abgebrochen wurde, die \textit{OnImageCreated} Methode aufgerufen.

\begin{lstlisting}
    public class ParallelImageGenerator : AsyncImageGenerator
    {
        private readonly object _bitmapLock = new object();
        private readonly object _currentRowLock = new object();
        private int _currentRow;
        private Bitmap _bitmap;

        protected  override void BuildBitmap(Area area)
        {
            var stopwatch = new Stopwatch();
            stopwatch.Start();
            _currentRow = 0;
            _bitmap = new Bitmap(area.Width, area.Height);
            var threads = new Thread[Settings.DefaultSettings.Workers];

            for (int i = 0; i < threads.Length; i++) {
                Thread t = new Thread(() => BuildBitmap(area, Source.Token));
                threads[i] = t;
                t.Start();
            }

            foreach (var thread in threads) {
                thread.Join();
            }
            stopwatch.Stop();
            if (Source.Token.IsCancellationRequested)
                _bitmap = null;
            OnImageGenerated(area, _bitmap, stopwatch.Elapsed);
        }


        private void BuildBitmap(Area area, CancellationToken cancellationToken)
        {
            while (!cancellationToken.IsCancellationRequested)
            {
                var row = GetCurrentRow();
                if (row >= area.Height)
                    return;

                for (int column = 0; column < area.Width; column++) {
                    var color = ImageGeneratorLogic.GetColorForData(area, column, row, cancellationToken);
                    lock (_bitmapLock) {
                        _bitmap.SetPixel(column, row, color);
                    }
                }
            }
        }

        private int GetCurrentRow() {
            int row;
            lock (_currentRowLock) {
                row = _currentRow;
                _currentRow++;
            }
            return row;
        }
    }
\end{lstlisting}

\subsection{d. Performancevergleich}
Wie in den Anforderungen definiert wurde der Vergleich mit folgenden Parametern durchgeführt:

\begin{itemize}
	\item MaxIterations: 10000
	\item ZBorder: 4
	\item MaxImg: -0,02
	\item MaxReal: -1,32
	\item MinImg: -0,1
	\item MinReal: -1,4
	\item Workers: 10
\end{itemize}

\begin{table}[!htb]
\centering
\caption{Performancevergleich: Ergebnisse}
\begin{tabular}{cccc}
\hline
                                                                           
Run & Synchron [ms] & Parallel [ms] & \\
1 & 4955 & 1561 & \\
2 & 4958 & 1432 & \\
3 & 5010 & 1512 & \\
4 & 5077 & 1498 & \\
5 & 4993 & 1532 & \\
6 & 4981 & 1554 & \\
7 & 4941 & 1388 & \\
8 & 4959 & 1451 & \\
9 & 4939 & 1523 & \\
10 & 5026 & 1486 & \\ 
Mittelwert    & 4983 & 1493 & \\
Std. Abw.    & 41,5      & 52,6  &\\
Speed Up & & & 3,34

\end{tabular}
\end{table} 

Dieser Vergleich zeigt, dass durch ein Parallelisieren der einzelnen Berechnungen ein enormer Speed Up erreicht werden kann.

\end{document}